\documentclass[UTF8]{ctexart}
	\title{wpf框架的变革要素}
	\author{st.krwlng}
\begin{document}
	\maketitle
	\tableofcontents
	\subsection{控件树结构}
	wpf结构化地重构了控件的视觉结构与逻辑结构,其核心观点之一是控件本身也是由控件组成的(窗体本身也是控件),以最大化地增强控件的可重用性,也因此使得wpf程序的控件对象以树形结构联系在一起。wpf引擎基于其控件树结构,提供了其他一些机制以增强控件的可复用性与可扩展性。
		\subsubsection{模板机制}
		模板机制为wpf下游程序员提供了接口,使其能够访问并修改(包括标准控件在内的)上游wpf控件的布局和样式。下游程序员可以任意限度地修改上游控件的布局和样式,而继续复用该控件的界面逻辑——只要在布局中定义了界面逻辑所需的控件对象(并保持x:Name属性一致)。这样达成了控件的样式与界面逻辑的解耦。
		\paragraph{样例1} 想要Button控件为圆角矩形,只需要重写Button的模板,使得其原本内部的Border控件为圆角矩形即可。
		\paragraph{样例2} 想要另ComboBox右边的按钮变成圆角矩形,需要先重写ComboBox的模板,在其模板内部重写ToggleButton的模板,将ToggleButton的外观变为圆角矩形即可。

		早期的UI框架都没有为下游程序员专门提供修改控件的布局与样式的接口,因此想要实现较复杂的自定义样式,往往需要重新实现一个控件,在定义它的布局与样式的同时也要重写一个完全相同的界面逻辑,因此增加了时间成本。
		\subsubsection{路由事件机制}
		路由事件为构成父子关系的控件提供了新的消息传递机制,在控件树上相距较远的控件之间可以通过路由事件传递消息,从而减少了参数传递的成本(如果不使用路由事件,控件树底层的控件需要得到高层控件的引用,因此需要以方法参数的形式不断传递下去)。此外,由于wpf中控件本身包含控件,因此不可避免地会有控件在视图中重叠。为了能在编程中清晰地定义响应事件的控件,大部分事件以路由的形式由控件树叶节点向根部传递(比如鼠标事件)。
	\subsection{依赖属性与数据绑定}
	数据绑定机制提供了新的界面元素属性赋值方法,减少了显式为界面元素属性赋值与得到反馈所需的代码、以及相关事件处理方法的数量;同时,该架构对MVVM模式的开发更加友好。

	依赖属性除了为数据绑定机制服务,还提供了基于控件树的参数传递机制。将依赖属性定义为“可继承的”,则可使子控件继承祖先控件相同属性的值。
\end{document}
