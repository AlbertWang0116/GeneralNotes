\documentclass[UTF8]{ctexart}
    \title{Maven创建Java类库项目}
    \author{st.krwlng}
\begin{document}
    \maketitle
    \tableofcontents
    \subsection{创建命令与目录结构} 使用mvn archetype:generate可以在当前目录下创建基本的java工程(phase: archetype、goal: generate)。在创建的过程中会要求输入groupId,该参数与java源码中得package等效;artifactId,该参数为项目名称,也是创建的目录名称;version,项目的版本。groupId、artifactId、version构成的三元组唯一描述了一个Maven工程。项目创建完成后出现一个{artifactId}名称的目录,该目录下有pom.xml文件,以及目录src/main/java/{groupId}和src/test/java/{groupId}。其中,src/main/java/{groupId}目录下存放项目源码,maven在编译过程中会编译所有该目录下的java源码文件到target/classes/{groupId}下;src/test/java/{groupId}存放该项目的单元测试源码文件,maven在单元测试阶段会编译并执行该目录下的所有java源码文件。pom.xml描述了maven的构建规则,对于Java类库项目,构建规则中比较重要的标签有groupId(编译阶段要编译的文件的位置由根目录+groupId确定,其中根目录默认为src/main/java/)、artifactId、dependencies、plugins。
    \subsection{依赖库描述} pom.xml文件的dependencies标签描述了需要从本地或服务器上获取的依赖库文件。依赖库可以从远端服务器加载,在这种情况下,dependency标签下主要需要描述的标签为groupId、artifactId和version。根据该三元组,Maven在编译前会先尝试获取dependency标签描述的依赖库,具体获取顺序是:1、查询本地缓存目录中是否有所需依赖库文件;2、如果未获取到,会尝试从Maven中心服务器下载该依赖库文件;3、如果未获取到,尝试从本地自定义的远程服务器下载依赖库文件(在pom.xml的repositories标签下可以自定义其他远程服务器);4、如果未获取到,则依赖库获取失败,项目构建终止。只有所有dependencies描述的外部依赖库文件都成功获取才会开始编译阶段。依赖库除了从外部获取,也可以指定本地路径并让maven从本地加载,此时dependency的子标签会有所变化,groupId、artifactId为本地依赖库文件名,version可以为任意。较重要的标签为scope,需要为system;systemPath,依赖库在本地机器上的文件系统路径。
    \subsection{编译插件描述} 生成项目工程时pom.xml中已经存在一个artifactId为maven-compiler-plugin的插件,该插件为默认的编译插件。该插件完成了在源码目录下自动搜索java源码文件并编译的过程。plugin标签下的configuration子标签用于设置编译命令时的参数,其中有source,相当于javac的-source参数;target,相当于javac的-target参数;debug,该参数默认为true,此时编译有-g参数;compilerArguments下可以设置其他自定义参数,比如javac的-bootclasspath参数可以在这里设置,方法是添加bootclasspath为compilerArguments的子标签,并将值设置为bootclasspath标签的内容。
    \subsection{单元测试} 生成项目工程时pom.xml中已经存在junit依赖库,在执行mvn test时,maven会自动将测试源码目录内的单元测试文件编译并用junit执行。maven可以自动生成单元测试覆盖率的报告,不过需要在配置文件中添加额外的插件:在plugins标签下添加groupId为maven、artifactId为maven-clover-plugin、version为2.4.2(可能已经有新版本)的plugin。添加完毕后,在单元测试时使用命令mvn cobertura:covertura,单元测试完成后可以在target/site/cobertura/index.html中查看单元测试覆盖率的情况,以及未被覆盖的分支和代码。
    \subsection{maven初步认识} maven的运行生命周期是一个固定顺序的phase序列,maven本身只是一个构建框架,行为全部由plugin决定。每个plugin可以绑定到maven生命周期的一个或多个phase中,并创建一个或多个goal,并在每个goal中运行特定的脚本。
\end{document}
